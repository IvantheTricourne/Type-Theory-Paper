\title{Martin-L\"of Type Theory: Example proof}
\author{Carl Factora}
\date{\today}

\documentclass[12pt]{article}
% The following packages are needed because unicode
% is translated (using the next set of packages) to
% latex commands. You may need more packages if you
% use more unicode characters:
\usepackage{amssymb}
\usepackage{bbm}
\usepackage[greek,english]{babel}
% This handles the translation of unicode to latex:
\usepackage{ucs}
\usepackage[utf8x]{inputenc}
\usepackage{autofe}

% Some characters that are not automatically defined
% (you figure out by the latex compilation errors you get),
% and you need to define:
\DeclareUnicodeCharacter{8988}{\ensuremath{\ulcorner}}
\DeclareUnicodeCharacter{8989}{\ensuremath{\urcorner}}
\DeclareUnicodeCharacter{8803}{\ensuremath{\overline{\equiv}}}
\DeclareUnicodeCharacter{120738}{\ensuremath{\Sigma}}
\DeclareUnicodeCharacter{955}{\ensuremath{\lambda}}

% Add more as you need them (shouldn’t happen often).
% Using “\newenvironment” to redefine verbatim to
% be called “code” doesn’t always work properly. 
% You can more reliably use:
\usepackage{fancyvrb}

\DefineVerbatimEnvironment
  {code}{Verbatim}
  {} % Add fancy options here if you like.

\begin{document}
\maketitle

\section*{Preliminary definitions}

\begin{code}
  open import Data.Nat using (ℕ; suc)
  open import Relation.Binary.PropositionalEquality
    using (_≡_; refl; cong; sym; trans)
  open import Data.Product
  open import Data.Empty

  -- Natural Number Recursion and Induction principle
  recℕ : (C : Set) → C → (ℕ → C → C) → ℕ → C
  recℕ C c f 0 = c
  recℕ C c f (suc n) = f n (recℕ C c f n)

  indℕ : (C : ℕ → Set) → C 0 → ((n : ℕ) → C n → C (suc n)) →
         (n : ℕ) → C n
  indℕ C c f zero = c
  indℕ C c f (suc n) = f n (indℕ C c f n)

  -- Addition
  add : ℕ → ℕ → ℕ
  add = recℕ (ℕ → ℕ) (λ n → n) (λ m r n → suc (r n))
\end{code}
We also define the following propositions:
\begin{code}
  ¬ : Set → Set
  ¬ A = A → ⊥

  _≤_ : (i j : ℕ) → Set
  i ≤ j = 𝞢[ k ∈ ℕ ] (add i k ≡ j)

  _<_ : (i j : ℕ) → Set
  i < j = (i ≤ j) × ¬ (i ≡ j)

  0≠si : (i : ℕ) → ¬ (0 ≡ suc i)
  0≠si i = λ ()

\end{code}
Here, we define {\tt ≤} and {\tt <} as:
\begin{itemize}
\item {\tt i ≤ j} is provable if and only if there exists a {\tt k} such that
  {\tt i + k = j}.
\item {\tt i < j} is provable if and only if we can provide a proof that
  {\tt i ≤ j} and that {\tt i ≠ j}.
\end{itemize}
Thus, both propositions are proven by constructing by pairs. The proof for
{\tt i ≤ j} must be a dependent pair since its corresponding proposition
uses a specific natural number (viz. {\tt k}).

{\tt 0≠si} is a proof that 0 is never equal to the successor of a natural
number. Here, we write {\tt λ ()} since the second assumption of the proof,
{\tt 0 ≡ suc i}, allows us to simply appeal to the fact that it is absurd to say
that 0 is equal to any natural number {\tt + 1}, by definition of the natural
numbers and the fact that Agda assumes that terms that are constructed
differently (i.e., 0 and {\tt suc} for natural numbers) can never be equal.

\section*{The Proof}
Using the above, we can prove the following:
\begin{code}
  0< : (i : ℕ) → (0 < suc i)
  0< = ?
\end{code}
The proof of this proposition requires that we prove that 0 is less than all
natural numbers. We do this using induction. More specifically, this proof
requires us to perform induction on a natural number, {\tt i}. To do
this, we use the {\tt indℕ} principle defined above. For each step of our proof,
we also provide the corresponding proof environment (viz. {\tt Proof Environment})
that represents the given proof obligation for each hole (viz. {\tt \{ \}}).
\begin{code}
0< : (i : ℕ) → (0 < suc i)
0< = indℕ { }0 { }1 { }2

{---Proof Environment
  ?0 : ℕ → Set
  ?1 : 0 < suc 0
  ?2 : (n : ℕ) → 0 < suc n → 0 < suc (suc n)
-}
\end{code}
The first of these obligations (i.e., {\tt ?0}) requires us to declare the
proposition we are wanting to prove. In general, this simply mirrors the proofs
type definition.
\begin{code}
0< : (i : ℕ) → (0 < suc i)
0< = indℕ (λ i → 0 < suc i) { }1 { }2

{---Proof Environment
  ?1 : 0 < suc 0
  ?2 : (n : ℕ) → 0 < suc n → 0 < suc (suc n)
-}
\end{code}
Our next obligation requires us to prove the proposition in the case that
\mbox{{\tt i = 0}} (i.e., we must show that {\tt 0 < suc 0}). Given our
definition of {\tt <}, the proof this by constructing a pair, containing proofs
of \mbox{{\tt 0 ≤ suc 0}} and \mbox{{\tt ¬ (0 ≡ suc 0)}}.

\begin{code}
0< : (i : ℕ) → (0 < suc i)
0< = indℕ (λ i → 0 < suc i)
          ({ }0 , { }1)
          { }2
  
{---Proof Environment
  ?0 : 0 ≤ suc 0
  ?1 : ¬ (0 ≡ suc 0)
  ?2 : (n : ℕ) → 0 < suc n → 0 < suc (suc n)
-}
\end{code}
To prove {\tt ?0}, we must construct a dependent pair that contains a natural
number {\tt k}, such that {\tt 0 + k = 1}. Thus, we construct a pair containing
the natural number 1 and as well as the proof that {\tt 1 = 1} (e.g. {\tt refl}).
To prove {\tt ?1}, we use our proof of {\tt 0≠si} on 0 to prove that
{\tt ¬ (0 ≡ suc 0)}.
\begin{code}
0< : (i : ℕ) → (0 < suc i)
0< = indℕ (λ i → 0 < suc i)
          ((1 , refl) , 0≠si 0)
          { }2
  
{---Proof Environment
  ?2 : (n : ℕ) → 0 < suc n → 0 < suc (suc n)
-}
\end{code}
To prove the final proposition, we must prove that for all natural numbers, {\tt n},
with the assumption that 0 is less than {\tt suc n}, we can show that 0 is less
than {\tt suc (suc n)} (i.e., {\tt n + 2}). 
\begin{code}
0< : (i : ℕ) → (0 < suc i)
0< = indℕ (λ i → 0 < suc i)
          ((1 , refl) , 0≠si 0)
          (λ n ih → { }0)

{---Proof Environment
  ?0 : 0 < suc (suc n)
-}
\end{code}
In this case, our inductive hypothesis, {\tt ih}, is actually not necessary,
since we can simply employ the same strategy as in the earlier proof for the
base-case.
\begin{code}
  0< : (i : ℕ) → (0 < suc i)
  0< = indℕ (λ i → 0 < suc i)
            ((1 , refl) , 0≠si 0)
            (λ n _ → ((suc (suc n)) , refl) , 0≠si (suc n))

\end{code}
\end{document}
